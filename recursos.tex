\chapter{Recursos avançados para linguistas}

Finalmente, chegamos ao ponto alto deste manual. Se não é o ponto alto, ao menos é o foco. Doravante vamos conhecer o poder do \LaTeX{} aplicado na Linguística.

\section{TIPA}

\texttt{TIPA} (Tokyo IPA) é um pacote criado por Fukui Rey para processar os caracteres do \emph{International Phonetic Alphabet}, que é usado para transcrições fonéticas. Suporta até mesmo os caracteres considerados obsoletos pelas revisões de 1989 e 1996.

Para usá-lo, devemos adicionar a linha \verb+\usepackage{tipa}+ o preâmbulo do documento. Para usar o alfabeto fonético, escrevemos o comando \verb+\textipa{ }+. Por exemplo:

\begin{center}
\emph{Input:} \verb+textipa{/kaza/}+

\emph{Output:} \textipa{/kaza/}
\end{center}

Ou algo mais sofisticado, como:

\begin{center}
\emph{Input:} \verb+\textipa{[""Ekspl@"neIS@n]}+

\emph{Output:} \textipa{[\textsecstress Ekspl@"neIS@n]}
\end{center}

O pacote \texttt{TIPA} resolve muitos problemas. O primeiro deles é que é muito lento escrever símbolos fonéticos no \emph{Word}, por exemplo: você tem que selecionar os símbolos, um por um, dentro de uma grande tabela.

Já com o \texttt{TIPA}, escrevemos seguindo uma convenção. O \emph{output} segue esta convenção, que é descrita numa tabela muito parecida com a tabela clássica do IPA, que você pode ver \href{http://www.ling.ohio-state.edu/events/lcc/tutorials/tipachart/tipachart.pdf}{neste link} (\textsf{http://www.ling.ohio-state.edu/events/lcc/tutorials/tipachart/tipachart.pdf}).

Um conjunto de letras \emph{sans-serif} vem junto com o \texttt{TIPA}, acessíveis via \verb+\textsf{\tex-tipa{ }}+. Podemos usar, ainda, o ambiente 

\begin{center}
\begin{minipage}[h]{1cm}
\verb+\begin{IPA}+

\vdots

\verb+\end{IPA}+
\end{minipage}
\end{center}

para escrever longos trechos de transcrição fonética.

O pacote \texttt{TIPA} inclui o pacote \texttt{vowel}, que permite desenhar a \textbf{tabela de vogais}:

\begin{center}
\begin{minipage}[h]{6cm}
\begin{verbatim}
\begin{vowel}
	\putcvowel{i}{1}
	\putcvowel{e}{2}
	\putcvowel{\textipa{E}}{3}
	\putcvowel{a}{4}
\end{vowel}
\end{verbatim}
\end{minipage}
\begin{minipage}[h]{6cm}
\begin{vowel}
        \putcvowel{i}{1}
	\putcvowel{e}{2}
	\putcvowel{\textipa{E}}{3}
	\putcvowel{a}{4}
\end{vowel}
\end{minipage}
\end{center}

(A tabela de vogais acima está incompleta.) Para usar o ambiente \texttt{vowel}, você deve incluir, no preâmbulo, a linha \verb+\usepackage{vowel}+.

O manual do pacote \texttt{TIPA} está no seguinte endereço: \href{http://uit.no/getfile.php?PageId=874\&FileId=165}{\textsf{http://uit.no/getfile.php?PageId=874\&FileId=165}}

Slides de um curso muito completo disponíveis no site do \emph{Center for Advanced Study in Theoretical Linguistics}, da \emph{Universitetet i Troms\o} podem ser vistos no link \href{http://uit.no/getfile.php?PageId=874\&FileId=299}{\textsf{http://uit.no/getfile.php?PageId=874\&FileId=299}} 

\section{Árvores}

As ávores são usadas na Linguística, Lógica, Matemática, Ciências da Computação e outros campos. Existem softwares específicos para se criar árvores, mas eu não sei como eles funcionam. Achei muito mais simples, porém, usar o pacote \texttt{qtree}. Para usá-lo, você deve digitar, no preâmbulo, \verb+\usepackage{qtree}+. Vejamos como funciona uma árvore em \LaTeX:

\begin{center}
\begin{minipage}[h]{6cm}
\begin{verbatim}
\Tree [.S
	[.NP
		[.AR Aquele ]
		[.N homem ]
	]
	[.VP
		[.V comprou ]
			[.SN
			[.Ac duas ]
			[.N bolas. ]
		]
	]
]
\end{verbatim}
\end{minipage}
\begin{minipage}[h]{6cm}
\Tree [.S
	[.NP
		[.AR Aquele ]
		[.N homem ]
	]
	[.VP
		[.V comprou ]
			[.SN
			[.Ac duas ]
			[.N bolas. ]
		]
	]
]
\end{minipage}
\end{center}

Para compor uma árvore, basta um pouco de lógica. Ela sempre se inicia com o comando \verb+\Tree+. A hierarquia é definida por colchetes ``[''. Veja uma árvore mais simples:

\begin{center}
\begin{minipage}[h]{6cm}
\begin{verbatim}
\Tree [.A
	[.B
		[.C
			D
			E
			F
		]
		[.G ]
	]
]
\end{verbatim}
\end{minipage}
\begin{minipage}[h]{6cm}
\Tree [.A
	[.B
		[.C
			D
			E
			F
		]
		[.G ]
	]
]
\end{minipage}
\end{center}

Para um passo-a-passo sobre como fazer árvores, recomendo novamente os slides do \emph{Center for Advanced Study in Theoretical Linguistics}: \href{http://uit.no/getfile.php?PageId=874\&FileId=299}{\textsf{http://uit.no/getfile.php?PageId=874\&FileId=299}}, a partir da página 18.

\section{\emph{Attribute-value matrices}}

Usadas na gramática gerativista, as AVMs são outro desafio para a representação. O \LaTeX{} supre isso muito bem. Basta chamar o pacote AVM no preâmbulo do documento, usando o comando \verb+\usepackage{avm}+ e desenhá-la como abaixo:

\begin{center}
\begin{minipage}[h]{8cm}
\begin{verbatim}
\begin{avm}
\[
category & \emph{noun phrase} \\

agreement & \[ number \emph{singular} \\
	       person \emph{third}
	    \]
\]
\end{avm}
\end{verbatim}
\end{minipage}
\begin{minipage}[h]{6cm}
\begin{avm}
\[
category & \emph{noun phrase} \\

agreement & \[ number \emph{singular} \\
	       person \emph{third}
	    \]
\]
\end{avm}
\end{minipage}
\end{center}
