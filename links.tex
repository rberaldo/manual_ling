\chapter{Links}

Estes são os links recomendados por mim, ou que consultei para fazer este trabalho:

\begin{itemize}
	\item Site de um evento sobre \LaTeX{} na Linguística durante a primavera de 2005 na \emph{Universitetet i Troms\o}. Possui slides e \emph{handouts} do curso: \href{http://uit.no/castl/4727/}{{\sf http://uit.no/castl/4727/}}
	\item \emph{``Ten typographic mistakes everyone makes''}, por Christopher Phin: \href{http://www.recedinghairline.co.uk/files/c1c3be2fda2b218e858029a4bde7e96c-397.html}{{\sf http://www.recedinghairline.co.uk/fil\-es/c1c3be2fda2b218e858029a4bde7e96c-397.html}}
	\item \emph{``Using \LaTeX{} on Windows''}: \href{http://www.pinteric.com/miktex.html}{{\sf http://www.pinteric.com/miktex.html}}
	\item \emph{Detexify\textsuperscript{2} --- LaTeX symbol classifier}: o \LaTeX{} provê centenas de símbolos, mas decorar todos os comandos é difícil. Desenhe o símbolo neste site e ele te dará o respectivo comando: \href{http://detexify.kirelabs.org/classify.html}{{\sf http://detexify.kire\-labs.org/class\-ify.html}}
	\item \emph{\LaTeX4Ling}, site muito interessante do \emph{Department of Language \& Linguistics} da \emph{University of Essex}, com muitos recursos e explicações: \href{http://www.essex.ac.uk/linguistics/external/clmt/latex4ling/}{{\sf http://www.essex.ac.uk/linguistics/external/clmt/latex4ling/}}
	\item \emph{``\LaTeX{} for Linguists''}, texto de Sebastian Nordhoff: \href{http://home.medewerker.uva.nl/s.nordhoff/page1.html}{{\sf http://home.medewerker.uva.nl/s.nordhoff/page1.html}}
	\item ``\LaTeX{} para Linguistas'', tradução do texto de Nordhoff por Rafael Luis Beraldo: \href{http://linguistica.ca\-ba\-la\-da\-da.org/2009/09/latex-para-linguistas/}{{\sf http://linguistica.ca\-ba\-la\-da\-da.org/20\-09/09/latex-para-linguistas/}}
	\item Site sobre \LaTeX{} do \emph{Department of Linguisics} da \emph{University of Pennsylvania}: \href{http://www.ling.upenn.e\-du/advice/latex.html}{{\sf http://www.ling.upenn.e\-du/ad\-vi\-ce/latex.html}}
	\item \emph{``\LaTeX{} for linguistics notes''}, texto do mesmo teor que este, porém mais completo: \href{http://www.wepapers.com/Papers/52000/Latex\_for\_\_Linguistics\_Notes}{{\sf http://www.we\-papers.com/Pa\-pers/52000/La\-tex\_for\_\_Linguistics\_Notes}}
\end{itemize}
