\chapter{Introdução}

O \LaTeX{} (pronunciado \textipa{/"leItEk/}) é uma \textbf{\gls{linguagem_marcacao}} muito usada por matemáticos, físicos e cientistas da computação. Para eles, o princial atrativo é a facilidade da composição de fórmulas matemáticas; em verdade, o \LaTeX{} é o padrão internacional\footnote{Isto é, \LaTeX{} é uma convenção na tipografia de fórmulas matemáticas.} de equações matemáticas em \href{http://en.wikipedia.org/wiki/Plain_text}{\emph{plain text}}. Enquanto os programas \href{http://en.wikipedia.org/wiki/WYSIWYG}{WYSIWYG} (acrônimo de What You See Is What You Get) focam em como o documento ficará, à semelhança das máquinas de escrever, o \LaTeX{} te dá a possibilidade de organizar seu documento logicamente em seções, dividí-lo em vários arquivos, e usar ferramentas avançadas.

O foco deste manual são justamente essas ferramentas. Na primeira parte, apresento uma série de razões (``filosóficas'') para o uso de \LaTeX{} pelos linguistas; depois falo de alguns tópicos de \textbf{tipografia}, tradição que foi praticamente soterrada com a popularização do \emph{Microsoft Word} e semelhantes; finalmente, me foco nos recursos para linguistas.

Acredito que essa organização seja necessária por dois motivos:

\begin{itemize}
	\item Dentro da Linguística encontrei muito pouco uso de \LaTeX. As únicas iniciativas que vi são internacionais. Acredito que uso de uma ferramenta científica em detrimento daqueles que são, na prática, amadoras, é completamente justificável e traria melhorias;
	\item Há muito pouca gente nas Humanas que conheça a filosofia do uso dos computadores, o que implica em um uso e visão insatisfatórios. A maioria se vale de soluções-padrão, que não seguem a ``tradição de uso'' do computador; a tentativa de ``tornar fácil'' uma ferramenta que é, \emph{per se}, difícil, causa muitos problemas que não estão sendo discutidos. Provavelmente isso não seja culpa dos usuários: o único caminho que conhecem é este. Introduzo uma filosofia do uso de \LaTeX{}, que se liga muito fortemente à filosofia do software livre em geral, como uma \emph{introdução a um novo modelo de uso e de pensamento};
\end{itemize}

Haveria um terceiro motivo se fôssemos levar em consideração as inconsistências causadas pelo uso das normas da ABNT, mas prefiro citá-los vagamente e ocasionalmente a desviar o foco deste e causar indisposições com os defensores e usuários da ``norma''.

Para aprender \LaTeX{}, sugiro que você use o sistema operacional \gls{gnulinux}\footnote{Por um motivo simples: instalá-lo no Linux é extremamente trivial.} e leia este documento: \href{http://tobi.oetiker.ch/lshort/lshort.pdf}{\emph{The not too short introduction to \LaTeXe{}}} ({\sf http://tobi.oetiker.ch/lshort/lshort.pdf}) ou a versão em português, aqui: \href{http://www.tug.org/texlive/Contents/live/texmf-doc/doc/portuguese/lshort-portuguese/ptlshort.pdf}{\emph{Uma não tão pequena introdução ao \LaTeXe}} ({\sf http://www.tug.org/texlive/Contents/live/texmf-doc/doc/portuguese/lshort-portuguese/ptlshort.pdf}).

Qualquer explicação extra que você precise pode ser encontrado na Internet. A maioria das dúvidas que você terá já foi respondida, especialmente se são dúvidas elementares, de iniciante. Por isso eu recomendo fortemente que aquele que deseje usar o computador como ferramenta procure conhecer as listas de e-mail, os repositórios de documentação, fórums, canais de \gls{irc}. Isso será útil durante seu aprendizado de \LaTeX{} e qualquer outro programa de código aberto.

Agora podemos começar.
