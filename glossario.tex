\glossarystyle{longheader}
\renewcommand*{\glossaryheader}{
\bfseries Entrada & \bfseries Descrição\\\endhead
}

\newglossaryentry{linguagem_marcacao}{
	name={linguagem de marcação},
	description={é um conjunto de códigos aplicados a um texto ou a dados, com o fim de adicionar informações particulares sobre esse texto ou dado, ou sobre trechos específicos, segundo a \href{http://pt.wikipedia.org/wiki/Linguagem_de_marcação}{Wikipédia}},
	plural={linguagens de marcação}
}

\newglossaryentry{gnulinux}{
	name={\textsc{gnu}/Linux},
	description={sistema operacional \emph{open source}, baseado no kernel Linux e aplicativos abertos do projeto \textsc{gnu}}
}

\newglossaryentry{irc}{
	name={\textsc{irc}},
	description={protocolo de comunicação bastante usado na Internet}
}

\newglossaryentry{compilacao}{
	name={compilação},
	description={compilador é o programa de computador que interpreta um código fonte e cria a versão final, em uma linguagem que o computador entenda. ``Compilação'' é o processo de criar essa versão final usando-se o programa. No mundo do \LaTeX, porém, o processo de compilação significa transformar do código que você digita em \texttt{PDF} ou qualquer outro formato desejado},
	plural={compilações}
}

\newglossaryentry{hifenizacao}{
	name={hifenização},
	description={é o processo de colocação de hífens, de modo a justificar o texto tentando manter uma quantidade razoável, nem muito grande nem muito pequena, de caracteres na linha. Pressupõe um dicionário completo na memória do computador, o que dá trabalho e custa caro. As distribuições de \LaTeX, porém, trazem dicionários de hífen sem nenhum custo. Isso garante que a qualidade tipográfica do documento será muito superior àquela obtida por meio do \emph{Word}, por exemplo, que somente adiciona ou remove espaços entre os caracteres para conseguir justificar o texto},
	plural={hifenizações}
}
